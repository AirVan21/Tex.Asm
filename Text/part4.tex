\section{Лекция 4}
\subsection{Выделение памяти}
Загружает программу в себя, перехватывает прерывание, печатает трассу. 
Другой вариант - программа, в которой есть breakpoint. 
\begin{rem}Нужно ставить CC, а не int 3.\end{rem}

Трассировка - пошаговое выполнение. Загружаем именно .com файл. Имя вызываемой программы - hardcode. Сначала написан сегмент дебагера, далее сегмент загруженной программы. Наблюдали, что сработало прерывание и напечатало нам строку. Трассировки подвергается каждый сегмент. 

Выделение памяти? Alloc (или Malloc) выделение памяти в куче. Каждой запускаемой программе выделяется 4 Гб памяти. Т.е. мы можем обратиться к любому выделенному байту. 

mov al, адрес

Говорим про виртуальной адресное пространство. Это поддерживается процессором. Процессор не проецирует адресное простарнство на физическую память. Только после запроса адреса процессор разбирается, в какой физической памяти лежит требуемая ячейка. При обращении к неспроецированной памяти - разница в выполнение - большая. 

Обработчик 14-го прерывания. Появляется в режиме protected. Выделение памяти - проецирование запрашиваемой памяти на физичекскую. Но это не так. Нужно отдельное усилие, чтобы спроецировать память на физическую. Сброс происходит страницами по 4Кб. Квант сброса на диск.  

\begin{rem} Жестко привязать - (non-paging pool) некоторые связки виртуальной и физической памяти никогда не разрушаются. Запрос на пямять, который мы не можем удовлетворить. Должны заниматься сбросом\end{rem}


Trash - для обслуживания обращения мы часто рвем одни и те же свзи, выделяя ресурсы. Алгоритмы не идеальны. Порой нет связи, которую можно порвать. Чем больше жесткой привязки, тем больше trash.
Программисты решаю, когда использовать жесткую привязку памяти. С проблемой проецирования памяти столкнулся обработчик прерывания 14 (32 битный режим). Его нужно привязать жестко.

Есть таблицы для определения проекций, которыми пользуется процессор.

DDOS - Denial of Service. Всегда при установлении сокет в TCP/IP всегда немного выделяется в памяти non-paging pool. Поэтому возможна атака.

\begin{off}Защищаться от воков. Охотиться на волков.\end{off} 

\subsection{Выделение памяти в однозадачной системе}
Как работатет DOS. Сидит внизу, выделяет память под PSP (Program Segment Prefix), загружает код программы. Потом выделяет всю память под нашу программу. Norton Guide сидел в памяти до того как мы загрузили нашу программу. NG активизировался, получив управление, но структура памяти не поменялась. Управление у Norton Guide, но у резидента только та память, которую он сначала себе отрезал.

Вся память у последнего исполняемого процесса. Он её обрезает, можно в получившейся свободной памяти запустить процесс. Теперь все оставшая память снова принадлежит последнему исполняемогу процессу. Уже смотрели то, как "отрезают" память.

Говорим про системные Alloc. Трехслойнай архитектура Alloc'ов. Процесс свободный, но должен ходить к ОС и отмечаться. "Я намерен обращаться к законно принадлежащему участку".

Триумвират - обсуждения alloc-malloc от Ромы, Артура и Баклановского.
 
Вызвыать функцию 4Bh - Load and Execute. Не в своем адресном пространстве.

Мы должны работать в своем адресном пространстве. Мы не режем новый кусок. Мы запускаем файл для трассировки в памяти, которою операционка считает истинное нашей. Пока мы не "резанулись" выделять нечего. 

Пишем план:
\begin{enumerate}
\item Загрузить "файл 1" в память. Неприятности - выравнивание align. Начало PSP - program segment prefix. Нужно поставить за 100h. Взять нулевое смещение - это не получится. Сегмент может начаться, если у нас последний 0.
Мы должны загружаться с выровненной точки
\item Передать управление "файлу 1".
\item Поймать возврат управления (позаботиться об это заранее).
\item Перехват / восстановление вектора V1.
\item Обработчик прерывания 1 - печать трассы.
\item Печаталка есть. Нужно аккуратно pop/push. Посмотреть, что происходит в современных компиляторах. Посмотреть пролог и эпилог. Есть процессорные инструкции. Организация работы со стеком.
\item Включить трассировку. Выключить трассировку.
\item Breakpoint - ставим, куда хотим, но дебагером. (CC - затирает, но нужно восстановить). Если мы хотим продолжить, то это нормально.
\end{enumerate}

Главный пункт, взять и передать управление. 

%(buf - _ + 100h + 15)/16;

%(buf - _ + 100h + 15)/16*16;

%(buf - _ + 100h + 15)/16*16 + 100h;

Поймать возврат - вещь из 2-х частей. По ret. Управление приходит в начало PSP, туда можно что-то поставить и в стек положить.

6-й включить/выключить trace flag. Одна инструкция.

Table 6-1. Исключения и прерывания.

Режимы работы процессора
\begin{enumerate}
\item Real Mode. 16-bit. Все стартует в Real Mode. 
\item Protected Mode. 32-bit.
\item 64-bit Mode. Extended memory mode.
\end{enumerate}

AMD перывае разработали 64-битную архитектуру. EM64T - Extended Memory 64 Technology - это расширение. Не новая архитектурная разработка. Появился этот 64-битный режим. 
\begin{rem}В Real Mode интересует первые 8 штук. Частый вопрос на экзамене.\end{rem}
Инструкция UD - Undefined, придуманная для генерации exception'а.

Загрузили. Как передать управление.

в PSP -> куда файл залили. 

CS => PSP
IP => в 100h

retF CS:IP => пункт 3

jmp PSP : 0100h

call PSP : 100h

push cs

push ip

retf


Рома рассказывает про то, как поймать возврат. 