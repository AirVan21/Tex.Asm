\section*{Debugger}
\begin{enumerate}
\item Должны загрузить файл, который будем анализировать в память. Создаем файл hw1.asm. В нем находится код, который выводит строку на экран.

File handle - номер, который ОС временно назначает для файла, когда его открывает.

Ставим, что считываем \bfseries FFFF\mdseries байт, т.к. уверены, что считав столько данных мы сможем взять всю программу, т.к. COM файл имеет ограничение в 64Кб.

Мы последняя запущенная программа. Нам принадлежит вся свободная оперативная память. Чтобы опредить, куда загрузить анализируемую программу достаточно посчитать, сколько место занимает текст основной программы и учесть 100h под параметры. Для вычисление "веса" тела программы ставим метку в конце, а в начале у нас уже есть. 

Зачем 0 после fname в программе? Без него на загружает.

Вычисляем размер программы в параграфах, которые равны 16 байт.
Заметили, что в dump по смещению 280 уже лежит наша программа.

Кладем по смещению 180 - CD.

Убил DOS неправильным ret'ом.

Для режима трассировки действуем прерыванием 1.
\end{enumerate}