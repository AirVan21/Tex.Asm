\section{Лекция 9}
\begin{rem} Баклановский говорит, что на экзамене будет специально запутывать. Стоит смотреть, с чем соглашаться, а с чем нет.\end{rem}
\subsection{Видео}
Смотрим разные решения для видео: mono - 2 цвета, CGA(Color)- 4 цвета, EGA(Extended), VGA(Video), SVGA(SuperVideo). GA - это Graphic Adapter(Graphical Array) - для нас в названиях нет особой разницы. VGA  поддерживается современными устройствами. Ранее те, кто писал в порты адаптера, дублировали запись в память. Это необходимо, чтобы другие узнали информацию о видео-режиме, что мы сделали. Т.к. до VGA порты адаптера только на запись. B8000 - адрес видео буфера. Дублирование принято из-за старых моделей. Достаточно в памяти поменять форму курсора. И курсор становится именно таким, каким нам нужно. С VGA пришло 256 цветов - это была революция. В EGA была подготовительная работа для VGA.

Пирамида абстракций. По X - возможности. По Y - абстракции. Идя вверх по абстракциям, мы лишаемся огромного количества возможностей. 

Что ещё пишется в память - сами порты. 3D и 3B (посмотрели symdeb,-dw0:463 - первый байт 03D4).

Процессор как-то общается с устройством(in,out). В устройстве за этим реально стоит 2 ячейки, т.е. 2 порта. Один индексный, другой - порт данных. Внутри куча регистров и указатель. Вся куча регистров адресуется через указатель. Сначала говорим, какую ячейку хотим записать. Заполняем значением индекс. Потом передаем данные. По индексу записываем полученные данные. Шина - интерфейс, обертка (ISA, PCI).


Через порт 3D4 будем посылать адрес. Через 3D5 посылаем данные. Пишем в sumdeb:

mov dx, 3d4

mov al, a

out dx, al

inc dx

mov al, 1  ; вроде, так

out dx, al

dec dx

mov al, b

out dx, al

inc dx

mov al, .. ; что-то нужно положить

out dx, al ;

press enter



Сначала послали индекс, потом данные. Если кто-то влезет между нашими out, выполняющими данную задачу, то это может плохо закончиться. Нет атомарности, хотя действие должно проходить, как транзакция. Для этого, на данном этапе, можно исправить все, используя \bfseries cli, sti.\mdseries

Буквы адаптер рисовать не умеет. Весь экран разбит на ячейки 8x25. Ячейки состоят из пикселей. В прошлый раз мы заполняли коды. Каждому коду соответсовало заполнение. Таблица такая - знакогенератор. Появилась в EGA. В VGA можно и читать эту таблицу.
\begin{hw}Постотреть таблицу знакогенератора\end{hw}
ASCII - соответствие кода и картинки. Знакогенератор - соответствие кода и заполнения ячейки пикселяеми. Проблема со шрифтами - они плохо масштабируются.

Курсор - это линии.

\subsection{PCI}
Little Endian: mov x, ax; в памяти получим x = al | ah. Именно так уложены. В al будет лежать адрес, в ah будут лежать данные. Одновременная передача индекса и данных. Избавились от возможной проблемой случайного прерывания. 

mov dx, 3d4

mov ax, 10a

out dx, ax

mov ax, 100b

out dx, ax

g(последняя ячейка)

У PCI 32-х битные порты. Мы сразу принимает 4 байта. Так идут все операции. 

Видео начало набирать мощь. Эффект белого листа. Экспериментально пытаются отличить лист бумаги от монитора. После 100 Гц. Люди перестают отличать. 
Получается ещё один класс устройств, которые захотел скоростной доступ к памяти. Видео уходит из I/O. Local Bus от фирмы Vesa. Intel предлодила PCI шину.

На шине между CPU и RAM ставили мезонин - bus. Вклиниваем шину. К этой шине стали присоединять монитор. Локальная шина - шина, имеющая непостредственный контакт с процессором и памятью. Рисунок с PCI шиной, которая соединяет много устройств(в центре PCI из неё идут пути к процессору, RAM, I/O). AGP шина идет в обход PCI, напрямую от видеокарты в RAM.

Картинка.
MCH - Memory Control Hub(соединяет CPU, RAP, AGP4X, ICH2). ICH2 соединяет в сеть ATA, USB, PCI, LAN. Но для установления взаимодействия должен возникнуть канал. Канал должен администрировать процессор. Есть вариант с CPU, но это прошлый век. Должны быть отдельные процессоры, как мы смотрели процессор DMA для взаимодействия точка-точка память-диск. Сетевая технология - туннель поверх Hub.

Чипсет - реализация архитектуры. 

\begin{defn}FCB - front size bus. Шина, напосредственно к процессору(от какого-нибуль switch)\end{defn}

Intel Core уже имеент вместо 2-х хабов - отдин switch. Каналы стали динамическими. 

На PCI не так много устройств можно посадить. Зато в PCI есть возможность делать еще шины PCI. Соединение связывающее PCI и PCI - называется PCI Bridge. Соединение 2-х разных сетей - bridge.

Адрессация географическая. Куда воткнули, отттуда и получился адрес. В устройстве не прописан адрес. Адрес в географии, там, куда подключаем. Структура адреса: Шина-Устройство-Функция-Регистры. Нумерации шин PCI. Устройство - номер. Внутри самого устройства поддерживаются нумерация функций. Все навороченные чипсеты поддерживают логическую адресацию PCI. 

Пример функции: в самом чипсете несколько устройств определены в один девайс. Спецификация доступна по адресу PCISIG - тусич про PCI. Как RFC.

Ограничение на PCI 512 МБ/сек (недостяжимая) - максимальная частота. 256, 128. Зависит от разрядности шины.

Нужна будет фолксономия для решения усложненной задачи. Коды нужно будет превратить в строчку. Нужно текстовый файл превратить в удобный. Нужно проиндексировать.  А потом искать.

47 страница документации. Configuration space.

Что было до PCI? На что претендует устройство? Устройство может претендовать на память, IRQ, порты, каналы DMA. Конфликт устройств может быть(могу претендовать на одни и те же наборы). Диспетчер должен выдвать, по идее. Устройство может иметь разные наборы претензий. Устройство никак не может узнать о том, занято ли, что нам нужно или нет. Поэтому нам предложиди jumper'ы.
Plag \_ Play - безумная надстройка. В PCI уже сразу был Plug\_Play.  

Холодный старт, потом POST, потом загрузка. На этапе POST происходят разводка портов. Формат адреса. Старший байт 80. На bus - 8 бит. device на одну bus - 5 бит. 8 функций на device.

\begin{hw}Комментарии по задаче. Нужно определить все PCI устройства и вывести их номера на экран. Для талантов - нужно составить соответствие номеров и имен и печать имена PCI устройств. 
Читаем по 4 байта из PCI. Multifunctional устройства нужно определить и отдельно поработать с ними. Перебираем функции устройства, только когда устройство multifunctional. Если VendorID = FFFF -> устройства нет. Если не FFFF, то печатать.
\end{hw}
    