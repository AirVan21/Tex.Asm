\section{Загрузка}
Смотрели, как искать BIOS устройств. Вспоминаем байт 55 AA. E9 - короткий jmp. EA - длинный jmp. После 55 AA - идет короткий jmp E9. 

Что происходит после включения компьютера? 
\begin{enumerate}
\item Процессор тестирует себя(большая микропорграмма, прошитая в процессор). 
\item Одно ядро идет на FFFF:0000. Real Mode.
\item Power On Self Test. Сначала идет тест оперативной памяти. Нужно, чтобы не было изменений в программе, которую мы пишем. Далее тест окружения (чипсет, менеджер памяти).  Установка векторов. Инициализация векторов(Поиск BIOS-ов устройств и их запуск(у них есть инициализационна программа).
\item Boot Strap Loader. Int 19h. CMOS - память, поддерживаемая батарейкой. Хранит в себе информацию о Real Time Clock, настройки, пароль от BIOS . IRQ 8. Int 70. Нам нужна информация config - информация о том, откуда мы будем грузиться. IO DOC поясняет, как работать с BIOS.  
\end{enumerate}

В CMOS написан пароль, который считывается программой BIOS. Из CMOS можно считать. Можно найти manual о том, как считать пароль BIOS из CMOS.

Если у нас RISC архитектура, то команда умножения - длинный наобр RISC инструкций. В CISC мы дешифруем любую команду в микрокод, который уже и исполняется непосредственно процессором.
Прошитая в процессор программа - набор сигналов на языке паяльника. 

Предполагаем, что грузимся с HDD.
С какого-то места(h7С0), первый сектор HDD диска записывается в память. BIOS в CMOS посмотрел, кто загрузочное устройство. Забрал один сектор с диска - 512 байт. Этих 512 байт хватает, чтобы считать ещё несколько байт.

MSDN AA - Microsoft открыли исходники XP, 2003 Server. 

Ядро современной Windows - под 100 МБ весит ядро.
\begin{hw}На что хватит одного сектора?\end{hw}
После того, как загрузилось ядро, можно указать, что грузить потом. Последовательность сектор-ядро-вся система. Между сектором и ядром - несколько инетересующих нас этапа.

Наша задача - грузиться с дискеты. Хотя грузимся с  дискеты, но всё равно смотрим диск. Пишем один сектор программы. Там есть поле про размер сектора. Если это поле обнулить, то даже с дискеты грузиться не будет. Защита диск Pro.

Мы для дискеты должны не убить служебные данные и написать корректно свои данные. Нас загрузит BIOS. И нужно показать вектора и область данных BIOS(500h-600h).

Вектора - точки входа в BIOS. Узнать то, что ещё ничего не меняла 
\subsection{Реализация}
Распечатали область данных (0-600h). Требование к дискете - она должна быть читаема, но должна быть и загружаема.

WinImage для создания загрузочной флешки.

Набросок структуры файловой системы. Как устроен FAT.
1 сектор - boot sector. Информация, что не загрузочная дискета, хранится на самой дискетете. Поэтому если, у нас не загрузочная флэшка, то мы распечатаем сообщение на экран с дискеты.

В boot-й сектор всегда записывается программа. Либо программа, которая печататет, что не загрузочная, либо, которая начинает загрузку.


FAT

FAT

ROOT

DATA

1 кластер = $2^n$ подряд идущих секторов. Сделаны кластеры для уменьшения адресации. Slack.
Адрессация лежит в FAT. Записи о файлах - название, размер, дата, ссылка на первую ячейку FAT.В первой ячейкой FAT - ссылка на первую ячейку FAT.

В FAT 16 битная ячейка. Между FAT и DATA - бикекция (сколько ячеек в FAT, столько и в DATA).
\begin{hw}
Какой размер FAT должен быть, чтобы натянуть на террабайтный диск.
\end{hw}  
32-х битная спецификация по FAT.
Жесткий диск делится на 4 partition. Каждый partition может быть подразбит на 4 partiotion'а.
Начинаются в первом секторе и идут деревом. Каждый такой partition может содержать файловую систему или дробится на другие patition? - что-то странное.

Форматирование - нанесения файловой системы. Разметка и создание разделов. 
В Disk Edit нужно будет просмотреть 

MBR - Master Boot Record - один сектор.

Ещё одни этам Boot Strap. MBR - тоже программа. MBR смотрит среди partition загрузочный и берет с него загрузочный. Нужно взять первый сектор и передать ему управление.