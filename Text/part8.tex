\section{Лекция 8}

\subsection{Доклады}
Нужно в мае подготовить доклад.

\begin{hw} Одна задача касается PCI. Нужно выдать коды PCI устройств. В следующий раз будем обсуждать PCI. Вторая задача - сделать свой загрузчик. Меняем boot сектор. Пишем свой и сами это складываем. И нужно с этого грузиться. Дополнительно к этой задаче нужно написать программу, которая будет показывать вектора и некоторую информацию.
\end{hw}

Обсуждаем темы. Все разделили на общеизвестные стандарты и процессорные команды. Темы разбирать будем в мае (или сейчас уже).

Memory <=> Процессор <=> I/O

Можно запрашивать у системы:
\begin{enumerate}
\item Порты (к I/O)
\item IRQ - дорога от устройства к осбобой ножке процессора
\item DMA - direct memory access. Устройства могут писать напрямую в память, не отвлекая процессор.
\item Память (memory)
\end{enumerate}

\begin{defn} Шина - среда передачи сигнала.\end{defn} 
Системная шина или локальная - шина, которая реализует физическую связь процессора и памяти. IDE - процессорная плата на диске. ATA/ATAPI - (PIO для записи в процессор). Внешние устройства подключаются к процессору через PIO, передавая данные которые будут проходить и через системную шину. 

Задача поддержки группы высокопроизводительных машин для мощных запусков. Постоянно были тесты железа. Появилась Win95. Жесткие диски стали работать в 2 раза быстрее. Удвоилась внешняя шина, обошли ограничения. Ultra DMA - могут появиться в лабораториях, но когда они добираются до пользователя - проходит время.

DMA предлагает прямой доступ к памяти. Процессор программирует DMA, говоря "что взять", "куда положить". Когда DMA заканчивает, оно должно оповестить процессор через IRQ.

SCSI - делал сразу DMA. Все всегда упирается в цену вопроса. Кому критично, покупали SCSI. Ранее не могли принять стандарт по ATA. Для каких классов устройства важен DMA?

I/O - можно тоже классифицировать. Это можно сделать самостоятельно. 

Как память на устройстве отображается на оперативную память. 

С BIOS'ом общаемся через порты. Через порты мы можем делать update.

Проецирование на память. Запрос "хочу mov из памяти в регистр". Сначала идет запрос на адрес в память. Если запрос на BIOS память, то BIOS не откликается, а используется информация из памяти.
Говорят, что "на память проецируются порты". Когда к BIOS'у обращаемся, как к памяти.   

\subsection{Boot}
Холодный reset - полный набор действий. Первый раз за много время производим запуск. Кнопкой reset.

Горячий reset - на лету, быстрая перезагрузка.

Адрес:\bfseries FFFF:0; \mdseries g105 - холодный reset. Адрес по которому перезагружаемся. Все это можем писать в symdeb.

top500.org сайт о суперкомпьютерах
Как стартует суперкомпьютер. Группа суперкомпьютеров подгатавливает к запуску этот суперкомпьютер. "Разогревает его".

Вообще изначально стартует одно ядро. Оно стартует в реальном моде.
Горячий reset это int18. 

Смотрим, как устроено все в памяти. После F000:0000 идет сам BIOS.

Итого получаем: вектора(1K), память BIOS(256b), DOS(до A000, т.е. 640К). Память на видеокарте проецируется на С000(32К). Можно посмотреть на неперехваченный int 10. Вообще это спроецировано, а значит - продублировано. 

Вообще, смотрим кусок 0 - 1 Мб. Особенные программы идут с устройствами и проецируются на память в "середину".

Стандарта в области жестких дисков нет. Не могут прийти к соглашению уже лет 30. Это же происходит с видео и сетевыми картами. 

Некоторая память может остаться свободной. 160К свободных. После половины С000 и до F000 располагается Upper Memory Block. Гарри Киллдел и его фирма Digital Research. DR-DOS Киллдела получил золотую медаль за лучшую ОС. Гейтс проиграл, но зимой он копирует DR-DOS в MS-DOS 5.1. Лицензионно чисто. Легче кодить, чем проектировать. Через 10 лет компания Киллдела перестала существовать. 

Есть ещё HighMemory. Нехватает битовых комбинаций. Больше чем $2^20$ адресовать нельзя. Когда 20-bit перестало быть физическим ограничениям, появились. 32-bit процессоры. Ошибка A20R. 

Было 20 ножек для передачи адреса. Теперь 24 ножки или 32. Должны эмулировать 20 битный режим. Должны обнулить те, что не используем. Intel отрезали с 21, а не 20. 

\begin{hw}Сколько памяти потеряли Intel(а нам добавило памяти)?\end{hw} Есть техника Родена. Есть неадрессуемые = теневые регистры. Когда из реального режимы в 32-bit'й, а потом обратно, то нам достаются под 16-битном режиме возможность использовать большую 32-bit'ю адресацию. Вопрос, как обратиться? Даст ответ на вопрос. 

Штрихованный: "от B000 до BFFF" под видеопамять. 

4000b на страницу. Чтобы быстро перемещаться между страницами - держим в памяти из 32К несколько страниц. Просмотр страниц.

Пишем на разные страницы. 
pn - резидент, показывающий страницы.
dm - display memory
\begin{hw}Взять утилиты для работы с DOS\end{hw}
ic - industrial computer

Как програмно определить, какой монитор? Ч.б или цветной?
Артур говорил про 61h, 7 bit.

\begin{hw}Попробовать в symdeb - поменять форму курсора через порты.\end{hw} 