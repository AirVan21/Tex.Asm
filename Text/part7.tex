\section{Лекция 7}
\subsection{Измерение времени}
Как можем узнать время? 02Ch - функция DOS. Тикает с частотой 18.2. Спрашивае время у DOS'а. 
Мучительная процедура вычиитания т.к. различные основания для минут, секунд, часов. Можно прочтитать из памяти BIOS'а по адресу 046Сh. Можно воспользоваться int 8, который прочитает значение по адресу 046Сh. Это удобно, т.к. достаточно будет вычитать из одного значения другое, чтобы определить временную разницу. 

Процессор меняет время в той ячейке, когда приходит 8 прерывание по IRQ 0.
Если переполнение, то 0470h - выставляется Time Overflow Flag. 
После того, как появилось 8-е прерывание, то 8-е прерывание генерирует 01Сh. Ранее DOS рапространял информацию по через int 28. Зачем нужно дополнительное прерывание 01Сh. Когда закончили обрабатывать прерывание, должны сообщить контроллеру прерыванию об этом.

Если мы сначала перехватим int 8, в начале своего обработчика позовем старый, потом вернемся, то после этого момента все происходит 01Сh.

Перехватили 8-й
\begin{enumerate}
\item Подготовить стек
\item Вызвать старый обработчик
\item Далее 1Сh
\end{enumerate}
Заметим, что 1Сh автоматом выполняет первые 2 действия. Мысль, что так будет удобнее. 

Есть ещё Real Time Clock. Можно опираться на него, по int 70h спрашивать. 

Можно спрашивать время у процессора, но это не всегда удобно.

Узнаем, что таймер 3-х канальный. 1 канал - важное. 0 канал - int 8. 2 канал - мы можем опрашивать.  
 
Чем отличается статическая память от динамической. Оперативная память - динамическая. Статическая (кэш) имеет большую скорость. Статическая помнит. Динамическая забывает. 

Для динамической памяти читают, перед тем, как все она забыла. Потом записывают снова. Рефреш оперативной памяти. Операцию производят для всей памяти. 

Вот для регенерации динамической памяти необходим 0-й канал таймера. Это реализовано аппаратно. Это находится внутри микросхемы памяти. Таймер 8253 - здесь 3 таймера = 3 канала, на 0-м канале (int 8 = System time). 2 - канал не стартует сам по себе, он отдам нам. 
\subsection{Пример работы с документацией. Реализация временной задержки}
Одно разовый проход по документации. Задачи - посторить временные задержки, меньшие, чем $\frac{1}{20}$ секунды. 

Нужны ли задержки в программе? Да, например, чтобы правильно писать в устройство(работа с устройствами по спецификации). Для реализации работы с пользователем - писать интерфейсы. Также необходимо, когда пишем тесты. 

В старых играх задержки зависили от частоты процессора (задержки были циклами). Поэтому на новых компьютерах все было очень быстро. Поэтому появились замедлители процессоров. 

Сделаем нормальную задержку. Поиск на F7. Смотрим на картинки, ожидая момент, когда становится понятно. Нужно, чтобы счетчик делал dec(), пока не получим 0. 27 \% документа io.  В листинге по таймеру ошибка. Неправильное управляющее слово.  Ошибка в последнем бите. 

Составляем самостоятельно управляющее слово. Alt + F1 - переключение между окнами.

\begin{hw}Почему тикает int 8, когда не дали отбой int 9. Ответ искать в разделе ПКП (раздел 3). Искать приоритеты. Как запретить прохождение сигнала от конкретного устройства? На уровне контроллера. Можно заблокировать все сигналы.\end{hw}

\begin{hw}Классификация прерываний от Intel. Нужно знать. INT 2 приходит снаружи на отдельную ногу. В эти exception'ы свалили сходии по природе вещи.\end{hw}

NMI идет из за ошибки в памяти
\begin{hw} Найти как замаскировать NMI\end{hw} 