\section{Лекция 1}
\begin{rem}Максим Викторович Баклановский.\end{rem}
Контакты:\bfseries baklanovsky\mdseries  : skype, @mail.ru, @gmail.com

Варианты организации работы:
\begin{enumerate} 
\item Загрузочная флешка.
\item Эмулятор(w32).
\item VM (Virtual Box).
\end{enumerate}
\begin{rem}Для эмулятора нужно попросить USBDOS.zip (по skype).\end{rem}

\begin{off}В будущем нужно будет сделать доклад по настоящим спецификациям. Intel. Гук - 9 ссылок. По легальным ресурсам. Длинная арифметика. Ларри Уолл - создатель Perl.\end{off}

\subsection{Работа с эмулятором}
Разархивируем в C. NC далее. NC.bat его на рабочий стол(там пути). MAX каталог. 
Источники информации Shift + F1 = Norton Guide(Help по asm).
Options DB = Подгружает базы. Справочник программиста. (Лукач, Гук). 

Ralf Brown - сбор информации, перерабатывает. Файл с ошибками процессоров. Каталог Ints c прерыванияеми (F4) - просмотреть. Cntl + B = функциональные кнопки. MultiEdit - редактор. (Alt + F4 = альтернативный редактор, F2  тут вызывате меню, Alt + F1 = список окон). F2 - свою меню в Norton Commander.

LU => IO.DOC читать тут. 

Util - утилиты. pepores.com резидент - программа на постоянное присутствие. "Взять вектора" и получают переодически . 

TSR - terminate and state resident. Можно посмотреть, где сидит Norton. Когда что-то из Нортона вызвали, то вся память у него, то он себя отрезает, ограничивая свой блок. Сюда загружается именно COMMAND.COM. Далее - свободная память, куда будут помещены наши программы.  

NG => Лукач => undocumented. 

Основные задачи - обработать прерывание с клавиатуры.   
 
"pepores" показал нам память(dump). Можно узнать, где лежит наш резидент. Нужно найти в память своей программы. 

Turbo ASM, macroASM. MASM. Ассемблеры, где TASM от Borland, MASM от Microsoft. 

DOS => Work => Shift + F4 = новый файл. Пробуем написать код.

model tiny ; чтоб не было сегментов.

.code
\begin{hw} 
org 100h ; Смещение в 256 байт, где хранятся параметры, а адресация сегмента кода, начинается с
		 ; 100h 
\end{hw}
here:
jmp start

m1 db 'Hello, ASM!', 13, 10,'доллар'

start: 

mov dx, offsset m1

mov ah, 9; когда вызываем прерывание, то передается управление бибилотчечной функции по адресу 21h

int 21h ; способ оформления библиотечной функции. MS-DOS call, который смотрит параметр из ah

ret 	; - почему процессорная(ассемблерная) команда ret завершает исполнение программы.
		; по сути ret возвращает управление, т.е. восстанавливает значение регистра команд их стека.

end here    

далее Shift + F1, print string - ему соответствуте int 21h прерывание + 9 в ah, а то, адрес того, что мы выводим должен помещаться в dx.

Вспомнить Рюриковича - сегментная адресация. F10 - выход их NG.

Сохранили и создаем объектный файл с именем "1"

\bfseries tams /m 1 \mdseries

\begin{rem}/? - узнать ключи. m - кол-во проходов. Написать программу, на которой ASM сделает столько проходов, сколько мы хотим.\end{rem} 

Проходим линковщиком

\bfseries tlink /x/t 1 \mdseries

.COM файл, который в отличие от .EXE файла имеет ограничения на размер(64 КБ). Как-то связано с 100h. 

Сделаем резидент программу 2.com - на постоянное присутствие в памяти. Примером резидет-программы является pepores. menu => mem (тут можно его  увидеть).

model tiny

.code

org 100h

m1:jmp start

m2:db 'Hello, I am TSR',13,10,'доллар'; shift F1 ASCII char

start: 

mov dx, offset m2; смещение считаем с нуля. 

mov ah, 9        ;

int 21h          ; передает управление на библиотеку

mov dx, 0        ; пишем и выходим. потом 12h

mov ax, 3100h    ;

int 21h          ;

end m1

\begin{hw}Найти pepores'm это место. Можно искать по тем символам-словам, которые есть в программе.\end{hw} 