\section{Лекция 3}
Организационные вопросы: дополнительная пара для проверки домашнего задания. 
\begin{hw}Задача: написать свой примитивный отладчик. 2 момента: реализовать трассировку и breakpoint'ы. Загрузить файл - будет проблемно. Выполняем функции операционной системы. 
\end{hw}
\begin{rem}
MS-DOS: Тим Патерсон.
DR-DOS: Гарри Килдалл (король 8-битных операционок). Ему IBM предлагали контракт, но он отказался, не согласившись подписать NDA.\end{rem}

Как открыть файл? MS-DOS переняла наследие от других ОС. Фредерик Брукс, руководивший разработкой первой ОС в современном смысле этого слова - IBM OS/360, получил Тьюринговскую премию. Shift + F1 - вызов справочной системы, там же искать функции для работы с файлами. 

OF(15) и рядом - устаревшие функции. File - ящик в картотеке, отсюда и handle - ручка этого ящика. Современные функции используют handle в качестве однозначного идентификатора открытого файла. Старый набор функций вынуждал хранить о файле значительно больше информации, чтобы открыть файл нужно было "описать ящик". Используем функции с 3Dh (60-го) номера. Есть основные handle: stdin, stdout, stderr (handle - это просто целое число). Обращаем внимание на зависимость от CF (для возврата информации об ошибки). JNC (jump no carry) удобная команда. Перекидываем handle в BX (посмотрели в close). Нужен будет READ, чтобы считать *.com-файл в память. Поиск длины файла: lseek (66-я функция) с конца и с нулевым смещением. Одно из возвращаемых этой функцией значение - текущая позиция в файле. Её и следует трактовать как размер файла в байтах. Эрик Рэймонд. После открытия нужно делать seek 3-го типа. Так узнаем длину файла. Можно длину указать (FF). Из диска - в память загрузились и больше не читаем. Нужно обработаться все завершения работы.
Ограничения: отлаживаемая программа .COM и завершается ret'м. Сам отладчик .COM.

Понадобится стэковый фокус. Прерывания - 2е дело (внутреннее дело). 

~//

Работаем в debugger'е.
Прыжок на 107, но 107 нет. в 108
90 - универсальная затиралка. Когда мы не ставим ключ /m => получили "nop" (Нет операции - такую инструкцию процессор успешно выбирает из памяти, декодирует, но ничего не делает. Только инкрементирует ip и переходит к следующей инструкции). Причина в том, что по умолчанию ассемблер совершает только один проход и за один проход ему не удаётся сгенерировать файл минимально возможного размера - из нескольких вариантов инструкций ему иногда приходится консервативно выбирать самый длинный. В некоторых случаях это приводит к тому, что ассемблер заменяет команду на более короткую, а свободные байты "забивает" nop-ами. Например, в 16-битном режиме существует как минимум два варианта безусловного перехода jmp: E9 + 2 байта на целое со знаком значение смещения, EB + 1 байт (для тех же целей). Если ассемблеру разрешён только один проход, он не может предполагать, что 1-байтового смещения окажется достаточно и резервирует под инструкцию jmp 3 байта. Если впоследствии окажется, что достаточно короткой инструкции, именно она и будет в итоге помещена в исполняемый файл. На место "лишнего" байта будет записан nop. Просто удалить лишний байт нельзя, так как изменятся смещения всех адресов и меток, расположенных ниже по тексту ассемблерного файла. Поправить их без лишнего прохода не получится.

С ключом /m ассемблер будет выполнять дополнительные проходы до тех пор, пока удаётся сокращать размер генерируемого исполняемого файла.
\begin{hw} Задача - сделать столько проходов, сколько мы задаем.\end{hw}

Дизассемблирует все подрярд. Процессор не понимает разницу между данными и инструкциями. Идет по логике линейного ассемблера. Для того, чтобы спозиционировать окно TurboDebugger (td) с дизассемблированным кодом в любую удобную позицию, можно воспользоваться хоткеем Ctrl+G и ввести нужный адрес.


\begin{rem}Говард Эйкен. Алан Тьюринг. 2 разных типа архитектуры: Гарвардская (адресные пространства кода и данных разделены), Принстонская (код и данные находятся в одном и том же адресном пространстве и неотличимы друг от друга). \end{rem}

0 в стеке положил DOS, чтобы ret'm перейти в начало сегмента и PSP, где записаны байты CD 20 (int 20h - завершение программы). Что лежит в стэке, чтобы мы вернулись. call делает call и кладет следующий за собой адрес.	

В начале сегмента лежит наша программа.

\begin{hw}Нужно стэком убить свой код. Чтобы стэк дошел до наших инструкций и затер код.\end{hw}

Golf - решить задачу, уменьшив размер программы. 

Комментарий к функции h4 печати 16-ричного знаения из файла 5.asm: ret - один, но возвращать он будет нас в разные места.

Как смотреть результат работы ассемблера без дебагера? 	Получаем листинг программы (tasm /m/l 5.asm).\slshape shl ax, 4 \upshape => разваливалась в 4 команды \slshape shr ax,1\upshape. Это некруто, так как ассемблер сделал что-то по своему усмотрению, не уведомив программиста. Написали .486 - разрешили использовать команды 486 процессора. Теперь, посмотрев листинг, видим, что у нас все заменилось на одну 3-х байтовую команду.

\subsection{Пишем прерывания}
Использование программных прерываний. Пишем свой код.
Пишем свою функцию для вывода строки.

Всего есть 256 векторов, это просто адреса обработчиков соответствующих прерываний, организованные в таблицу, расположенную по самым младшим адресам в памяти. Номер любого прерывания, таким образом - просто индекс в этой таблице. Младшие 8 - (0-7) процессорные исключения. Пример: 0-е исключение - это деление на 0. Процессор не может выполнить команду деления на ноль и с помощью механизма исключений (исключение - особый вид прерывания, значит, у него может быть обработчик, как у любого другого прерывания) делегирует решение этой проблемы программисту. Следующие 8 штук (8-F) аппаратные int'ы первого контроллера, 8 - таймер, 9 - клавиатура. (10h - 1Fh) BIOS'а прерывания, 13 - работа с диском, 16 - клавиатура. (20h - 2Fh) забрал себе DOS (21 - функции дос, 22, 23, 24 - некоторые хоткеи (вроде принудительного завершения текущей программы), 28 - недокументированное прерывание, содержащее функции, полезные для создания резидентных программ). Итого: 48 номеров. Посмотреть в Brown'e что осталось для пользователей. Это от (F0-FF) зарезервировано изначально.

iret выталкивает из стека 3 слова, а не одно, как обычный ret. iret - это способ выйти из обработчика прерывания, нужные значения в стек помещает инструкция int: IP, CS, FLAGS. 25, 35 поставить вектор, прочитать текущий вектор.
ds в .com-программах всегда равен cs и ss. Значение es не фиксируется, следует это помнить.  

Делаем диспетчеры и вызываем разные функции.
Shift + F3 = убрать выделение в редакторе nc.
Написали разные функции, вызываемые в зависимости от нашего условия. 

Гарри Киллдал. Создавал массив точек входа. Можно все тесты убрать. Диспетчеризация в начале. В самих функциях ничего не пишем.

mov dx, (offset start + 15)/16; расчёт размера программы в параграфах с верным округлением (необходимо для функции "завершить программу и оставить её резидентной в памяти). 

Ошибка будет. Relative quantity illegal. Тут требуется какая-то типизация. Смещение не рассматривает как смещение. Нужно вычесть какое-нибудь другое смещение, например, offset \_ = 100h, и добавить 100h.

Разделяли на клиентскую и резидентскую часть. Заполнилась колонка: взятые вектора (в отчёте утилиты TSR).

Посмотрели разработку сервисного разработичка прерываний. 
\begin{hw} Думать про задачу. Взять вектора: трассировка - первый int, breakpoint 3 - й int \end{hw}

Нужно грамотно проследить, куда возвращаться. Флаги для трассировки. Сначала перехватить int 1. Нужно отследить завершение программы. 
\begin{enumerate}
\item Загрузить программу в память. 
\item Сохранить регистры.
\item Установить флаги.
\item Печатать все IP.  
\item Когда ret, должны выйти.
\item Breakpoint на IP. 
\end{enumerate}
