\section{Лекция 10}
\subsection{Архитектура}
Процессор соединяется с "облачком", с которым соединены другие устройства. Говорим, что есть 2 канала: канал адреса, канал данных. Пример аналогия из сетей: siplex, half-duplex, full duplex.
Если все свести к одному канался - то это путь мультиплексирования. Нужно оценить все "за" и "против". Есть принципы фон Неймана: first draft. Тайминги инструкций современного Intel - таблица времен исполнения инструкций в тактах (Optimization manual, C). Некоторые инструкции исполняются за 0.3 или 0.5 тактов - внутренний параллелизм в процессоре. Никто не знает, за сколько выполнятся комады процессора. 4 млрд операций, 1 млрд тактов в секунду. Работа с памятью 10-16 Гбит/сек. Моучли и Эккарт - сконструировали "Эниак". Нейман опубликовал работу, подписанную только им, но описывающую работу целой группы. Нейман добавил от себя лично - от увлекался самомодифицирующими автоматами. Он продвинул идею того, что в памяти можно проводить модицикации памяти, чтобы получить самомодифицирующуюся программу. 

\begin{rem}
Гарвардцкая архитектура - разделенная память данных, память для адресов. Принстонская архитектура - совместная память. Проблема buffer overflow. 
\end{rem}

Способность модицикации программ - не увеличивает производительнотсть системы. Идея использования гарвардской архитектуры принесло с собой проблемы обращения с памятью и её модификация.

Принципы фон Неймана: 
\begin{enumerate}
\item Принцип двоичного кодирования
\item Принцип адресуемость памяти. Адресуемость блоками. Линейный порядок на множестве адресов. Идея RAM.
\item Принцип программного упралвения. Инстркции в памяти. Инструкции выполняются друг за другом.
\item Принцип одродности памяти.
\end{enumerate}

Смотрим шины: параллельные и последовательные. Internet - последовательные шины. Ethernet - последовательная шина. Кто быстрей? Необычный результат.

Раньше внутри PC были быстрей параллельные шины. Сейчас USB - последовательный. SATA последовательный. 

ISA, 

IBM PC/AT. Первая персоналка с HDD. Там появилось соединение ATA/IDE - идея обхода общей шины ISA, чтобы её не загружать. С видео - проблемы была решена проецированием в память. Работают все те же абстрацкии: через порты, через прерывания(ресурсы), через DMA, через память. Хитросплетение проводов. DOS не знает про USB, он посылает что-то в порты. 

\subsection{Программа PCI}
\begin{off}S.Warrent "Hacker's delights" - советуют читать.\end{off}

Win-image. Создаем дискету(3-х дюймовую). PCI1.COM - перетаскиваем туда. IMA - образ. Выбираем в DOS дискету(образ IMA). Так, наверное, можно будет передать данные с именами.

Порты, IRQ, отображенна память, DMA. То, что ведет нас к устройствам, как программистов. Что мы сейчас делаем? 

Записываем на HDD. Мы посылаем команды в контроллер диска. Это, все равно, использование по назначению.

Когда начинаем работать с конкретным PCI устройством, то снавала узнаем мета-данные.
Мы выясняем, что это сетевая карта, product ID, vendor ID.

Облако можно уподобить Internet. Наша IP абстракция - это 4-ка.

PCI-specification. (p.215)Configuration space.

Проблема.

Представим, что на разных заводов сделали 2 разных завода. Но они не согласовали, какие использовать порты. Одновременно тогда с этими устройствами рабоать на сможем. Основная проблема - согласовать все с производителями аппаратур(OS 360). Vendor'ы не захотели понять требования для включения в ОС. Майкрософты сами переписывают драйвера. Зная о том, что vendor'ы не договорятся, каждый вендор учить устройство поддерживать несколько диапазонов портов. Включаем устройства. Конфликтуют, нет? Если да, то джамперами можно поправить конфликт. Это не поток не поставить.

Придумали plug и play, на шине нельзя перевести все устройства в состояние конфигурирования. Т.е. мы хотим конфигурировать устройство, а устройство думает, что мы работаем не с ним. 

Как передать устройствам то, что мы в режиме конфигурации. У нас 2 разных дела: конфигурация иди типичная работа. 

\begin{rem}
PMP - экзотика. Архитектурно не предусмотрен режим конфигурирования. Поэтому фигачили всевозможные костыли. Модуль pmp - не проблема. Нужно было предусмотерть семантику switch. 
 \end{rem}
 
В PCI режим конфигурирования полностью отделен от обычного режима. В нашей задаче мы получаем доступ к конфигурационным данным. 

На всем этам фоне работают BIOS и устройства.

В следующий раз мы будем проходить от включения питания до запуска OS.
\begin{hw}Нужно готовить доклад, парень!\end{hw}

A сегмент, B сегмент - графическая память (отображения с видео - занято). Смотрим С сегмент.

FFFF:0000 - главный вход в BIOS. 

util -> mft (программа манифест)

В сегменте F000:0000 сидит главный BIOS. После того, как он  свои 64к обработал, то смотрим, какие ещё есть устройства(дополнительные).

55 AA - нужно искать пометку, что это BIOS устройства.(55 AA) просто нравилось.
Начиная с C000:0000  начинаем смотреть до F000:0000 по 4к.


В папке LU => bios. Роспись AA55h, читаем еще байт и делаем call. 

\begin{rem}Лёше привет!POST - power on self test.
\end{rem}

Программа инициалзации, которая идет, она узнает, что мы направили jumper'ами и все нужные значения проставит вектора на себы, САМО. 

\newpage

Программа. В dx порт. Потом в ecx кладем адрес + данные. Из ECX мы складываем в EAX,  т.к. посылатб можо только их EAX. Проверяем на -1. Если да, то перепрыгиваем печать.

Bus device function

Написали программу.
Нужно проверять флаг multifunctional. Проверяется по флагу.


